\documentclass[12pt,a4paper]{article}
\usepackage[top=2.5cm,bottom=2.5cm,left=2.5cm,right=2.5cm]{geometry}
\usepackage[english]{babel}
\usepackage[utf8]{inputenc}
\usepackage{xcolor}
\usepackage{setspace}

\setlength\parindent{0pt}

\begin{document}
\onehalfspacing
%\begin{flushright}
4 September 2020
%\end{flushright}

\vspace{.25cm}
To the editor,

\vspace{.5cm}

Please find attached our manuscript titled ``\emph{Adaptive dating and fast proposals: revisiting the
phylogenetic relaxed clock model}'' for consideration as a
research article in PLOS Computational Biology.

\vspace{.5cm}

 
In this paper we thoroughly explored the uncorrelated relaxed clock model, which is widely used in phylogenetic inference.
The objective of this analysis was to produce new MCMC operators for doing Bayesian phylogenetic inference under the relaxed clock model.
With the overwhelming availability of genomic data, fast and efficient phylogenetic methods are more important than ever.


\vspace{.5cm}
We explored three parameterisation methods and two cutting-edge proposal kernels, introduced an adaptive operator which learns the weights of other operators, and we introduced a family of tree operators which are specifically designed to work in a relaxed clock model setting. 
These methodologies were systematically benchmarked on a range of empirical datasets to assess their abilities at achieving convergence.

\vspace{.5cm}
This work has produced a setup which can explore relaxed clock space up to 65 times faster than previous setups, depending on the dataset.
The methods produced are very effective on large datasets.
Our results are open source and accessible via a user-friendly interface. 


\vspace{2cm}

%\begin{flushright}
Thank you for your consideration.


J. Douglas, R. Zhang, R. Bouckaert 

\end{document}