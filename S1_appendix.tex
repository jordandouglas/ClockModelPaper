\documentclass[12pt]{article}
\usepackage[utf8]{inputenc}

\usepackage{color,soul}
\usepackage{xcolor}
\usepackage{array}
\usepackage{mhchem}
\usepackage{mathtools}

\DeclarePairedDelimiter\ceil{\lceil}{\rceil}
\DeclarePairedDelimiter\floor{\lfloor}{\rfloor}

\bibliographystyle{plos2015}


\begin{document}


\section*{S1 Appendix: Rate quantiles}


\section{Linear piecewise approximation}

In this article we introduced a linear piecewise approximation of the i-cdf to improve the \textit{quant} method. Let $\hat{F}^{-1}(\mathcal{R}_i)$ be the piecewise approximation of $F^{-1}(\mathcal{R}_i)$. The approximation is comprised of $n$ pieces (where $n$ is fixed).


%$\hat{F}(\mathcal{R}_i)$ be the piecewise approximation of $F(\mathcal{R}_i)$, and let  be its inverse function. 

\begin{align}
\hat{F}^{-1}(\mathcal{R}_i) = F^{-1}(\floor{b_i}) + \begin{cases} \Big( F^{-1}(\floor{b_i} + 1) - F^{-1}(\floor{b_i}) \Big) \Big( b_i - \floor{b_i}  \Big) &\text{ if } \floor{b_i} < n-1  \\ 0 &\text{ if } \floor{b_i} = n-1 \end{cases}
\end{align}


where $b_i = \min \{ \max\{ \frac{n \times \mathcal{R}_i}{n-1}, \frac{n_0}{n-1} \}, \frac{n - 1 -n_0}{n-1} \}$ indexes $\mathcal{R}_i$ into one of the $n$ pieces $\floor{b_i}$. $n_0 = 0.1$ provides a lower and upper limits of the piecewise approximation, corresponding to $r_\text{min}$ and $r_\text{max}$ respectively. The lower and upper rate limits are equal to:



\begin{align}
r_\text{min} &= F^{-1}\Big( \frac{n_0}{n-1}\Big) = F^{-1}\Big( 0.001 \Big) \\
r_\text{max} &= F^{-1}\Big( \frac{n - 1 -n_0}{n-1}\Big) = F^{-1}\Big( 0.999 \Big),
\end{align}

when $n=101$ and $n_0=0.1$. It is important to ensure that any operators which act in rate space (as opposed to quantile space) respect these boundaries. The inverse of the approximation function (ie. the cdf $\hat{F}$) is 


\begin{align}
\hat{F}(r_i) = \begin{cases} \max \big(0, \frac{1}{n-1} \times \Big( v_i + \frac{r_i -  F^{-1}(\floor{v_i})}{F^{-1}(\floor{v_i}+1) - F^{-1}(\floor{v_i})} \Big) \big) &\text{ if } \floor{v_i} < n-1  \\
	1 &\text{ if } \floor{v_i} = n-1 
\end{cases}
\end{align}


where $v_i \in (0, 1, \dotso, n-1)$ is the piece which $r_i$ corresponds to:


\begin{align}
v_i = \max\limits_{j = 0}^{n-1} \{ j : \hat{F}^{-1}\big(\frac{j}{n-1}\big) < r_i \}.
\end{align}


As the piecewise approximation is linear, computing the derivatives of these two functions (required for computing Hastings ratios) are trivial:


\begin{align}
\frac{\partial}{\partial \mathcal{R}_i} \hat{F}^{-1}(\mathcal{R}_i) =  D \hat{F}^{-1}(\mathcal{R}_i) &= \begin{cases}  \Big(\hat{F}^{-1}(\floor{b_i}+1) - \hat{F}^{-1}(\floor{b_i})\Big) \times(n-1)  & \text{ if } \floor{b_i} < n-1 \\
0 & \text{ if } \floor{b_i} = n-1 \end{cases} \label{eqn:derivFinv}  \\
\frac{\partial}{\partial r_i} \hat{F}(r_i) = D \hat{F}(r_i) &= \frac{1}{\hat{F}^{-1}\Big(\hat{F}(r_i)\Big)} \label{eqn:derivF}
\end{align}





\section{Tree operators for rate quantiles}

Zhang and Drummond 2020 introduced several tree operators for the \textit{real} parameterisation -- including Constant Distance, Simple Distance, and Small Pulley. In this appendix, these three operators are extended to the  the \textit{quant} parameterisation.

Following the notation presented in the main article, let $t_i$ be the time of node $i$, let $0 < q_i < 1$ be the rate quantile of node $i$, and let $r_i = \hat{F}^{-1}(q_i)$ be the natural rate of node $i$.








\subsection{Constant Distance}



Let $X$ be a uniformly-at-random sampled internal node on tree $\mathcal{T}$. Let $L$ and $R$ be the left and right child of $X$, respectively, and let $P$ be the parent of $X$. Under the \textit{quant} parameterisation, the Constant Distance operator works as follows: \\


\textul{\textit{Step 1}}. Propose a new height for $t_X$:

\begin{align}
	{t_X}^\prime \leftarrow t_X + \alpha
\end{align}

where $\alpha \sim \text{ Uniform}(-w, w)$, for some window size $w$. Ensure that $\max\{t_L, t_R \} < {t_X}^\prime < t_P$, and if the constrant is broken then reject the proposal.  \\


\textul{\textit{Step 2}}. Recalculate $q_X$ as:



\begin{align}
	{q_X}^\prime  \leftarrow & \hat{F}\Big({r_X}^\prime \Big) \nonumber\\
				\leftarrow & \hat{F}\Big(\frac{t_P - t_X}{t_P - {t_X}^\prime} r_X \Big)\nonumber \\
				\leftarrow & \hat{F}\Big(\frac{t_P - t_X}{t_P - {t_X}^\prime} \hat{F}^{-1}(q_X) \Big).
\end{align}


This ensures that the genetic distance between $X$ and $P$ remains constant after the operation by enforcing the constraint:


\begin{align}
	r_X (t_P - t_X) = {r_X}^\prime (t_P - {t_X}^\prime).
\end{align}




\textul{\textit{Step 3}}. Similarly, propose new rate quantiles for the two children $C \in \{L, R\}$:



\begin{align}
	{q_C}^\prime  \leftarrow & \hat{F}\Big({r_C}^\prime \Big) \nonumber\\
				\leftarrow & \hat{F}\Big(\frac{t_X - t_C}{{t_X}^\prime - t_C} \times r_C \Big) \nonumber\\
				\leftarrow & \hat{F}\Big(\frac{t_X - t_C}{{t_X}^\prime - t_C} \times \hat{F}^{-1}(q_C) \Big).
\end{align}


Ensure that $r_\text{min} < {r_i}^\prime < r_\text{max}$ for all proposed nodes $i \in \{X, L, R\}$, and if the constraint is broken then reject the proposal. 





\textul{\textit{Step 4}}. Finally, compute the natural logarithm of the Hastings ratio as that of the determinant of the Jacobian matrix:






\begin{align}
	J &= \begin{bmatrix} \frac{\partial {t_X}^\prime}{\partial t_X} & \frac{\partial {t_X}^\prime}{\partial q_X} & \frac{\partial {t_X}^\prime}{\partial q_L} & \frac{\partial {t_X}^\prime}{\partial q_R} \\
			\frac{\partial {q_X}^\prime}{\partial t_X} & \frac{\partial {q_X}^\prime}{\partial q_X} & \frac{\partial {q_X}^\prime}{\partial q_L} & \frac{\partial {q_X}^\prime}{\partial q_R} \\
			\frac{\partial {q_L}^\prime}{\partial t_X} & \frac{\partial {q_L}^\prime}{\partial q_X} & \frac{\partial {q_L}^\prime}{\partial q_L} & \frac{\partial {q_L}^\prime}{\partial q_R} \\
			\frac{\partial {q_R}^\prime}{\partial t_X} & \frac{\partial {q_R}^\prime}{\partial q_X} & \frac{\partial {q_R}^\prime}{\partial q_L} & \frac{\partial {q_R}^\prime}{\partial q_R} \\ \end{bmatrix} \nonumber \\
			&= \begin{bmatrix} \frac{\partial {t_X}^\prime}{\partial t_X} & 0 & 0 & 0 \\
			\frac{\partial {q_X}^\prime}{\partial t_X} & \frac{\partial {q_X}^\prime}{\partial q_X} & 0 & 0 \\
			\frac{\partial {q_L}^\prime}{\partial t_X} & 0 & \frac{\partial {q_L}^\prime}{\partial q_L} & 0 \\
			\frac{\partial {q_R}^\prime}{\partial t_X} & 0 & 0 & \frac{\partial {q_R}^\prime}{\partial q_R} \\ \end{bmatrix}.
\end{align}


As $J$ is triangular, its determinant $|J|$ is equal to the product of diagonal elements:


\begin{align}
	\ln |J| =&  \ln \{ \frac{\partial {t_X}^\prime}{\partial t_X} \times \frac{\partial {q_X}^\prime}{\partial q_X} \times \frac{\partial {q_L}^\prime}{\partial q_L} \times \frac{\partial {q_R}^\prime}{\partial q_R} \} \nonumber\\
		=& \ln 1 + \ln \; D \hat{F}\Big( \frac{t_P - t_X}{t_P - {t_X}^\prime} \times \hat{F}^{-1}(q_X)  \Big) + \ln \frac{\partial}{\partial q_X}  \frac{t_P - t_X}{t_P - {t_X}^\prime} \hat{F}^{-1} \Big( q_X \Big) \nonumber\\
		& + \ln \; D\hat{F}\Big( \frac{t_X - t_L}{{t_X}^\prime - t_L} \times \hat{F}^{-1}(q_L) \Big) + \ln \frac{\partial}{\partial q_L} \frac{t_X - t_L}{{t_X}^\prime - t_L} \hat{F}^{-1} \Big( q_L \Big) \nonumber\\
		& + \ln \; D \hat{F}\Big( \frac{t_X - t_R}{{t_X}^\prime - t_R} \times \hat{F}^{-1}(q_R) \Big) + \ln \frac{\partial}{\partial q_R} \frac{t_X - t_R}{{t_X}^\prime - t_R} \hat{F}^{-1} \Big( q_R \Big) \nonumber\\
		=& \ln \; D \hat{F}\Big( \frac{t_P - t_X}{t_P - {t_X}^\prime} \times \hat{F}^{-1}(q_X)  \Big) + \ln \; D  \hat{F}^{-1} \Big( q_X \Big) + \ln \frac{t_P - t_X}{t_P - {t_X}^\prime} \nonumber\\
		& + \ln \; D\hat{F}\Big( \frac{t_X - t_L}{{t_X}^\prime - t_L} \times \hat{F}^{-1}(q_L) \Big) + \ln \; D \hat{F}^{-1} \Big( q_L \Big) + \ln  \frac{t_X - t_L}{{t_X}^\prime - t_L} \nonumber \\
		& + \ln \; D \hat{F}\Big( \frac{t_X - t_R}{{t_X}^\prime - t_R} \times \hat{F}^{-1}(q_R) \Big) + \ln \; D\hat{F}^{-1} \Big( q_R \Big) + \ln \frac{t_X - t_R}{{t_X}^\prime - t_R} .
\end{align}







The derivatives $D\hat{F}$ and $D\hat{F}^{-1}$ are readily computed under the linear piecewise approximation. As its final step, the operator returns $\ln |J|$. 




\subsection{Simple Distance}

While Constant Distance proposes internal node heights, Simple Distance operates on the root. Let $X$ be the root node and let $L$ and $R$ be its two children.

\textul{\textit{Step 1}}. Propose a new height for $t_X$:

\begin{align}
	{t_X}^\prime \leftarrow t_X + \alpha
\end{align}

where $\alpha \sim \text{ Uniform}(-w, w)$, for some window size $w$. Ensure that $\max\{t_L, t_R \} < {t_X}^\prime$, and if the constraint is broken then reject the proposal.  \\


\textul{\textit{Step 2}}. Propose new rate quantiles for the two children $C \in \{L, R\}$:



\begin{align}
	{q_C}^\prime  \leftarrow & \hat{F}\Big({r_C}^\prime \Big)  \nonumber \\
				\leftarrow & \hat{F} \Big( \frac{t_X - t_C}{{t_X}^\prime - t_C} \times r_C \Big) \nonumber \\
				\leftarrow & \hat{F} \Big( \frac{t_X - t_C}{{t_X}^\prime - t_C} \times \hat{F}^{-1}(q_C) \Big).
\end{align}




These proposals ensure that the genetic distance between $X$ and its children $C$ remain constant after the operation by enforcing the constraint:


\begin{align}
	r_C (t_X - t_C) = {r_C}^\prime ({t_X}^\prime - t_C).
\end{align}


Ensure that $r_\text{min} < {r_C}^\prime < r_\text{max}$, and if the constraint is broken then reject the proposal. 



\textul{\textit{Step 3}}. Return the natural logarithm of the Green ratio by calculating the determinant of the Jacobian matrix $J$.



\begin{align}
	J &= \begin{bmatrix} \frac{\partial {t_X}^\prime}{\partial t_X} & \frac{\partial {t_X}^\prime}{\partial q_L} & \frac{\partial {t_X}^\prime}{\partial q_R} \\
						\frac{\partial {q_L}^\prime}{\partial t_X} & \frac{\partial {q_L}^\prime}{\partial q_L} & \frac{\partial {q_L}^\prime}{\partial q_R} \\
						\frac{\partial {q_R}^\prime}{\partial t_X} & \frac{\partial {q_R}^\prime}{\partial q_L} & \frac{\partial {q_R}^\prime}{\partial q_R} \end{bmatrix} \nonumber \\
		&= \begin{bmatrix} \frac{\partial {t_X}^\prime}{\partial t_X} & 0 & 0 \\
						\frac{\partial {q_L}^\prime}{\partial t_X} & \frac{\partial {q_L}^\prime}{\partial q_L} & 0\\
						\frac{\partial {q_R}^\prime}{\partial t_X} & 0 & \frac{\partial {q_R}^\prime}{\partial q_R} \end{bmatrix}.
\end{align}



As $J$ is triangular, its determinant $|J|$ is equal to the product of diagonal elements:


\begin{align}
	\ln |J| =&  \ln \{ \frac{\partial {t_X}^\prime}{\partial t_X} \times \frac{\partial {q_L}^\prime}{\partial q_L} \times \frac{\partial {q_R}^\prime}{\partial q_R} \} \nonumber \\
			=& \ln \frac{\partial {t_X}^\prime}{\partial t_X} +  \ln \frac{\partial {q_L}^\prime}{\partial q_L} + \ln \frac{\partial {q_R}^\prime}{\partial q_R} \nonumber \\
			=& \ln 1  \nonumber\\
			&+ \ln \; D \hat{F}\Big( \frac{t_X - t_L}{{t_X}^\prime - t_L} \times \hat{F}^{-1}(q_L) \Big) + \ln \frac{\partial}{\partial q_L} \frac{t_X - t_L}{{t_X}^\prime - t_L} \hat{F}^{-1} \Big( q_L \Big) \nonumber\\
			&+ \ln \; D \hat{F}\Big( \frac{t_X - t_R}{{t_X}^\prime - t_R}\times \hat{F}^{-1}(q_R) \Big) + \ln \frac{\partial}{\partial q_R} \frac{t_X - t_R}{{t_X}^\prime - t_R} \hat{F}^{-1} \Big( q_R \Big) \nonumber\\
			=& \ln \; D \hat{F}\Big( \frac{t_X - t_L}{{t_X}^\prime - t_L} \times \hat{F}^{-1}(q_L) \Big) + \ln \; D \hat{F}^{-1} \Big( q_L \Big) + \ln \frac{t_X - t_L}{{t_X}^\prime - t_L}\nonumber \\
			&+ \ln \; D \hat{F}\Big( \frac{t_X - t_R}{{t_X}^\prime - t_R}\times \hat{F}^{-1}(q_R) \Big) + \ln \; D \hat{F}^{-1} \Big( q_R \Big) + \ln  \frac{t_X - t_R}{{t_X}^\prime - t_R}.
\end{align}


$D \hat{F}(x)$ and $D \hat{F}^{-1}(x)$ are readily computed from the linear piecewise approximation. As its final step, the operator returns $\ln |J|$. 


\subsection{Small Pulley}

Just like the previous operator, Small Pulley operates on the root. Let $X$ be the root node and let $L$ and $R$ be its two children. However, unlike Simple Distance, this operator alters the two genetic distances $d_L = r_L (t_X - t_L) = \hat{F}^{-1} (q_L) (t_X - t_L)$ and $d_R = r_R (t_X - t_R) = \hat{F}^{-1} (q_R) (t_X - t_R)$, while conserving their sum $d_L + d_R$.



\textul{\textit{Step 1}}. Propose new genetic distances for $d_L$ and $d_R$:

\begin{align}
	{d_L}^\prime & \leftarrow d_L + \alpha \\
	{d_R}^\prime & \leftarrow d_R - \alpha
\end{align}

where $\alpha \sim \text{ Uniform}(-w, w)$, for some window size $w$. Ensure that $0 < {d_L}^\prime < d_L + d_R$ and that $r_\text{min} < {r_C}^\prime < r_\text{max}$ for $C \in \{ L, R \}$, and if either constrant is broken then reject the proposal.  \\




\textul{\textit{Step 2}}. Propose new rate quantiles for the two children $L$ and $R$:


\begin{align}
	{q_L}^\prime &\leftarrow \hat{F}\big( \frac{{d_L}^\prime}{t_X - t_L} \big) \nonumber\\
				 &\leftarrow \hat{F}\big( \frac{\hat{F}^{-1} (q_L) (t_X - t_L) + \alpha}{t_X - t_L} \big) \\
	{q_R}^\prime &\leftarrow \hat{F}\big( \frac{{d_R}^\prime}{t_X - t_R} \big) \nonumber \\
				 &\leftarrow \hat{F}\big( \frac{\hat{F}^{-1} (q_R) (t_X - t_R) - \alpha}{t_X - t_R} \big).
\end{align}



\textul{\textit{Step 3}}. Return the natural logarithm of the Green ratio by calculating the determinant of the Jacobian matrix $J$.


\begin{align}
	J &= \begin{bmatrix} \frac{\partial {q_L}^\prime}{\partial q_L} & \frac{\partial {q_L}^\prime}{\partial q_R} \\
						 \frac{\partial {q_R}^\prime}{\partial q_L} & \frac{\partial {q_R}^\prime}{\partial q_R} \end{bmatrix} \nonumber \\
	 &= \begin{bmatrix} \frac{\partial {q_L}^\prime}{\partial q_L} & 0 \\
						 0 & \frac{\partial {q_R}^\prime}{\partial q_R} \end{bmatrix}
\end{align}


As $J$ is triangular/diagonal, its determinant $|J|$ is equal to the product of diagonal elements:



\begin{align}
	\ln |J| =&  \ln \{ \frac{\partial {q_L}^\prime}{\partial q_L} \times \frac{\partial {q_R}^\prime}{\partial q_R} \} \nonumber \\
			=& \ln  \frac{\partial {q_L}^\prime}{\partial q_L} + \ln \frac{\partial {q_R}^\prime}{\partial q_R}\nonumber \\
			=& \ln \; D \hat{F}\big( \frac{\hat{F}^{-1} (q_L) (t_X - t_L) + \alpha}{t_X - t_L} \big) + \ln \frac{\partial {q_L}^\prime}{\partial q_L}  \frac{\hat{F}^{-1} (q_L) (t_X - t_L) + \alpha}{t_X - t_L} \nonumber\\
			&+ \ln \; D \hat{F}\big( \frac{\hat{F}^{-1} (q_R) (t_X - t_R) - \alpha}{t_X - t_R} \big) + \ln \frac{\partial {q_R}^\prime}{\partial q_R}  \frac{\hat{F}^{-1} (q_R) (t_X - t_R) - \alpha}{t_X - t_R} \nonumber\\
			=& \ln \; D \hat{F}\big( \frac{\hat{F}^{-1} (q_L) (t_X - t_L) + \alpha}{t_X - t_L} \big) + \ln \; D \hat{F}^{-1} (q_L) \nonumber \\
			&+ \ln \; D \hat{F}\big( \frac{\hat{F}^{-1} (q_R) (t_X - t_R) - \alpha}{t_X - t_R} \big) + \ln \; D\hat{F}^{-1} (q_R) .
\end{align}


Thus, as its final step, the operator returns $\ln |J|$.





\bibliography{references}




\end{document}