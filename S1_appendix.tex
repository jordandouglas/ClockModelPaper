\documentclass[12pt]{article}
\usepackage[utf8]{inputenc}

\usepackage[margin=1in]{geometry}

\usepackage{color,soul}
\usepackage{xcolor}
\usepackage{array}
\usepackage{mhchem}
\usepackage{mathtools}

\DeclarePairedDelimiter\ceil{\lceil}{\rceil}
\DeclarePairedDelimiter\floor{\lfloor}{\rfloor}

\bibliographystyle{plos2015}


\begin{document}




\vspace*{0.2in}

% Title must be 250 characters or less.
\begin{flushleft}
{\Large
\textbf\newline{Adaptive dating and fast proposals: revisiting the phylogenetic relaxed clock model} % Please use "sentence case" for title and headings (capitalize only the first word in a title (or heading), the first word in a subtitle (or subheading), and any proper nouns).
}
\newline
% Insert author names, affiliations and corresponding author email (do not include titles, positions, or degrees).
\\
Jordan Douglas\textsuperscript{1,2*},
Rong Zhang\textsuperscript{1,2},
Remco Bouckaert\textsuperscript{1,2}
\\
\bigskip
\textbf{1} Centre for Computational Evolution,  University of Auckland, Auckland, New Zealand\\
\textbf{2} School of Computer Science, University of Auckland, Auckland, New Zealand\\
\bigskip


% Use the asterisk to denote corresponding authorship and provide email address in note below.
* jordan.douglas@auckland.ac.nz


\end{flushleft}




\section*{S1 Appendix: Rate quantiles}



\section{Piecewise linear approximation}

In this article we introduced a linear piecewise approximation of the i-CDF (inverse cumulative distribution function) to improve the computational performance of the \textit{quant} parameterisation. 
Let $\hat{F}^{-1}(\mathcal{R}_i)$ be the piecewise approximation of the i-CDF $F^{-1}(\mathcal{R}_i)$. 
The approximation consists of $n$ pieces (where $n=100$ is fixed).
Due to the nonlinear nature of small and large quantiles in a Log-normal distribution, the first and last pieces are not linear approximations but rather equal to the underlying distribution itself. 


%$\hat{F}(\mathcal{R}_i)$ be the piecewise approximation of $F(\mathcal{R}_i)$, and let  be its inverse function. 

\begin{align}
\hat{F}^{-1}(q) = \begin{cases} 
F^{-1}(q) &\text{ if } q \leq \frac{1}{n} \text{ or } q \geq \frac{n-1}{n} \\ 
F^{-1}(\floor{v}) + \Big(F^{-1}(\floor{v}+1) - F^{-1}(\floor{v}) \Big) \Big( v - \floor{v} \Big)  &\text{ otherwise}. 
\end{cases}
\end{align}

where $v = q(n-1)$ indexes quantile $q$ into piece number $\floor{v}$.
Values from the underlying function $F^{-1}$ are cached, enabling rapid computation.




\section{Tree operators for rate quantiles}


Zhang and Drummond 2020 introduced several tree operators for the \textit{real} parameterisation -- including \texttt{ConstantDistance}, \texttt{SimpleDistance}, and \texttt{SmallPulley} \cite{zhang2020improving}.
In this appendix, these three operators are extended to the \textit{quant} parameterisation.
Following the notation presented in the main article, let $t_i$ be the time of node $i$, let $0 < q_i < 1$ be the rate quantile of node $i$, and let $r_i = \hat{F^{-1}}(q_i)$ be the real rate of node $i$ where $\hat{F^{-1}}$ is the linear approximation of the i-CDF.








\subsection*{Constant Distance}



Let $\mathcal{X}$ be a uniformly-at-random sampled internal node on tree $\mathcal{T}$. Let $\mathcal{L}$ and $\mathcal{R}$ be the left and right child of $\mathcal{X}$, respectively, and let $\mathcal{P}$ be the parent of $\mathcal{X}$. 
Under the \textit{quant} parameterisation, the \texttt{ConstantDistance} operator works as follows: \\


\textul{\textit{Step 1}}. Propose a new height for $t_\mathcal{X}$:

\begin{align}
	{t_\mathcal{X}}^\prime \leftarrow t_\mathcal{X} + s\Sigma
\end{align}

where $\Sigma$ is drawn from a proposal transition distribution (Uniform or Bactrian), and $s$ is a tunable step size. Ensure that $\max\{t_\mathcal{L}, t_\mathcal{R} \} < {t_\mathcal{X}}^\prime < t_\mathcal{P}$, and if the constraint is broken then reject the proposal.  \\


\textul{\textit{Step 2}}. Recalculate $q_\mathcal{X}$ as:



\begin{align}
	{q_\mathcal{X}}^\prime  \leftarrow & \hat{F}\Big({r_\mathcal{X}}^\prime \Big) \nonumber\\
				\leftarrow & \hat{F}\Big(\frac{t_\mathcal{P} - t_\mathcal{X}}{t_\mathcal{P} - {t_\mathcal{X}}^\prime} r_X \Big)\nonumber \\
				\leftarrow & \hat{F}\Big(\frac{t_\mathcal{P} - t_\mathcal{X}}{t_\mathcal{P} - {t_\mathcal{X}}^\prime} \hat{F^{-1}}(q_\mathcal{X}) \Big).
\end{align}


This ensures that the genetic distance between $\mathcal{X}$ and $P$ remains constant after the operation by enforcing the constraint:


\begin{align}
	r_\mathcal{X} (t_\mathcal{P} - t_\mathcal{X}) = {r_\mathcal{X}}^\prime (t_\mathcal{P} - {t_\mathcal{X}}^\prime).
\end{align}




\textul{\textit{Step 3}}. Similarly, propose new rate quantiles for the two children $\mathcal{C} \in \{\mathcal{L}, \mathcal{R}\}$:



\begin{align}
	{q_\mathcal{C}}^\prime  \leftarrow & \hat{F}\Big({r_\mathcal{C}}^\prime \Big) \nonumber\\
				\leftarrow & \hat{F}\Big(\frac{t_\mathcal{X} - t_\mathcal{C}}{{t_\mathcal{X}}^\prime - t_\mathcal{C}} \times r_\mathcal{C} \Big) \nonumber\\
				\leftarrow & \hat{F}\Big(\frac{t_\mathcal{X} - t_\mathcal{C}}{{t_\mathcal{X}}^\prime - t_\mathcal{C}} \times \hat{F^{-1}}(q_\mathcal{C}) \Big).
\end{align}


Ensure that $0 < {q_i}^\prime < 1$ for all proposed nodes $i \in \{\mathcal{X}, L, R\}$, and if the constraint is broken then reject the proposal. 
This constraint can only be broken from numerical issues.





\textul{\textit{Step 4}}. Finally, in order to calculate the Metropolis-Hastings-Green ratio, return the determinant of the Jacobian matrix:






\begin{align}
	J &= \begin{bmatrix} \frac{\partial {t_\mathcal{X}}^\prime}{\partial t_\mathcal{X}} & \frac{\partial {t_\mathcal{X}}^\prime}{\partial q_\mathcal{X}} & \frac{\partial {t_\mathcal{X}}^\prime}{\partial q_\mathcal{L}} & \frac{\partial {t_\mathcal{X}}^\prime}{\partial q_\mathcal{R}} \\
			\frac{\partial {q_\mathcal{X}}^\prime}{\partial t_\mathcal{X}} & \frac{\partial {q_\mathcal{X}}^\prime}{\partial q_\mathcal{X}} & \frac{\partial {q_\mathcal{X}}^\prime}{\partial q_\mathcal{L}} & \frac{\partial {q_\mathcal{X}}^\prime}{\partial q_\mathcal{R}} \\
			\frac{\partial {q_\mathcal{L}}^\prime}{\partial t_\mathcal{X}} & \frac{\partial {q_\mathcal{L}}^\prime}{\partial q_\mathcal{X}} & \frac{\partial {q_\mathcal{L}}^\prime}{\partial q_\mathcal{L}} & \frac{\partial {q_\mathcal{L}}^\prime}{\partial q_\mathcal{R}} \\
			\frac{\partial {q_\mathcal{R}}^\prime}{\partial t_\mathcal{X}} & \frac{\partial {q_\mathcal{R}}^\prime}{\partial q_\mathcal{X}} & \frac{\partial {q_\mathcal{R}}^\prime}{\partial q_\mathcal{L}} & \frac{\partial {q_\mathcal{R}}^\prime}{\partial q_\mathcal{R}} \\ \end{bmatrix} \nonumber \\
			&= \begin{bmatrix} \frac{\partial {t_\mathcal{X}}^\prime}{\partial t_\mathcal{X}} & 0 & 0 & 0 \\
			\frac{\partial {q_\mathcal{X}}^\prime}{\partial t_\mathcal{X}} & \frac{\partial {q_\mathcal{X}}^\prime}{\partial q_\mathcal{X}} & 0 & 0 \\
			\frac{\partial {q_\mathcal{L}}^\prime}{\partial t_\mathcal{X}} & 0 & \frac{\partial {q_\mathcal{L}}^\prime}{\partial q_\mathcal{L}} & 0 \\
			\frac{\partial {q_\mathcal{R}}^\prime}{\partial t_\mathcal{X}} & 0 & 0 & \frac{\partial {q_\mathcal{R}}^\prime}{\partial q_\mathcal{R}} \\ \end{bmatrix}.
\end{align}


As $J$ is triangular, its determinant $|J|$ is equal to the product of diagonal elements:


\begin{align}
	\ln |J| =&  \ln \{ \frac{\partial {t_\mathcal{X}}^\prime}{\partial t_\mathcal{X}} \times \frac{\partial {q_\mathcal{X}}^\prime}{\partial q_\mathcal{X}} \times \frac{\partial {q_\mathcal{L}}^\prime}{\partial q_\mathcal{L}} \times \frac{\partial {q_\mathcal{R}}^\prime}{\partial q_\mathcal{R}} \} \nonumber\\
		=& \ln 1 + \ln \; D \hat{F}\Big( \frac{t_\mathcal{P} - t_\mathcal{X}}{t_\mathcal{P} - {t_\mathcal{X}}^\prime} \times \hat{F^{-1}}(q_\mathcal{X})  \Big) + \ln \frac{\partial}{\partial q_\mathcal{X}}  \frac{t_\mathcal{P} - t_\mathcal{X}}{t_\mathcal{P} - {t_\mathcal{X}}^\prime} \hat{F^{-1}} \Big( q_\mathcal{X} \Big) \nonumber\\
		& + \ln \; D\hat{F}\Big( \frac{t_X - t_\mathcal{L}}{{t_\mathcal{X}}^\prime - t_\mathcal{L}} \times \hat{F^{-1}}(q_\mathcal{L}) \Big) + \ln \frac{\partial}{\partial q_\mathcal{L}} \frac{t_\mathcal{X} - t_\mathcal{L}}{{t_\mathcal{X}}^\prime - t_\mathcal{L}} \hat{F^{-1}} \Big( q_\mathcal{L} \Big) \nonumber\\
		& + \ln \; D \hat{F}\Big( \frac{t_\mathcal{X} - t_\mathcal{R}}{{t_\mathcal{X}}^\prime - t_\mathcal{R}} \times \hat{F^{-1}}(q_\mathcal{R}) \Big) + \ln \frac{\partial}{\partial q_\mathcal{R}} \frac{t_\mathcal{X} - t_\mathcal{R}}{{t_\mathcal{X}}^\prime - t_\mathcal{R}} \hat{F^{-1}} \Big( q_\mathcal{R} \Big) \nonumber\\
		=& \ln \; D \hat{F}\Big( \frac{t_\mathcal{P} - t_\mathcal{X}}{t_\mathcal{P} - {t_\mathcal{X}}^\prime} \times \hat{F^{-1}}(q_\mathcal{X})  \Big) + \ln \; D  \hat{F^{-1}} \Big( q_\mathcal{X} \Big) + \ln \frac{t_\mathcal{P} - t_\mathcal{X}}{t_\mathcal{P} - {t_\mathcal{X}}^\prime} \nonumber\\
		& + \ln \; D\hat{F}\Big( \frac{t_\mathcal{X} - t_\mathcal{L}}{{t_\mathcal{X}}^\prime - t_\mathcal{L}} \times \hat{F^{-1}}(q_\mathcal{L}) \Big) + \ln \; D \hat{F^{-1}} \Big( q_\mathcal{L} \Big) + \ln  \frac{t_\mathcal{X} - t_\mathcal{L}}{{t_\mathcal{X}}^\prime - t_\mathcal{L}} \nonumber \\
		& + \ln \; D \hat{F}\Big( \frac{t_\mathcal{X} - t_\mathcal{R}}{{t_\mathcal{X}}^\prime - t_\mathcal{R}} \times \hat{F^{-1}}(q_\mathcal{R}) \Big) + \ln \; D\hat{F^{-1}} \Big( q_\mathcal{R} \Big) + \ln \frac{t_\mathcal{X} - t_\mathcal{R}}{{t_\mathcal{X}}^\prime - t_\mathcal{R}} .
\end{align}







The derivatives $D\hat{F}$ and $D\hat{F^{-1}}$ are computed using numerical approximations for the first and last pieces, or as the gradient of the linear approximation for internal pieces.
As its final step, the operator returns $\ln |J|$. 




\subsection*{Simple Distance}

While \texttt{ConstantDistance} proposes internal node heights, \texttt{SimpleDistance} operates on the root. Let $\mathcal{X}$ be the root node and let $\mathcal{L}$ and $\mathcal{R}$ be its two children.

\textul{\textit{Step 1}}. Propose a new height for $t_\mathcal{X}$:

\begin{align}
	{t_\mathcal{X}}^\prime \leftarrow t_\mathcal{X} + s\Sigma.
\end{align}

Ensure that $\max\{t_\mathcal{L}, t_\mathcal{R} \} < {t_\mathcal{X}}^\prime$, and if the constraint is broken then reject the proposal.  \\


\textul{\textit{Step 2}}. Propose new rate quantiles for the two children $\mathcal{C} \in \{\mathcal{L}, \mathcal{R}\}$:



\begin{align}
	{q_\mathcal{C}}^\prime  \leftarrow & \hat{F}\Big({r_\mathcal{C}}^\prime \Big)  \nonumber \\
				\leftarrow & \hat{F}\Big( \frac{t_\mathcal{X} - t_\mathcal{C}}{{t_\mathcal{X}}^\prime - t_\mathcal{C}} \times r_\mathcal{C} \Big) \nonumber \\
				\leftarrow & \hat{F}\Big( \frac{t_\mathcal{X} - t_\mathcal{C}}{{t_\mathcal{X}}^\prime - t_\mathcal{C}} \times \hat{F^{-1}}(q_\mathcal{C}) \Big).
\end{align}




These proposals ensure that the genetic distance between $\mathcal{X}$ and its children $\mathcal{C}$ remain constant after the operation by enforcing the constraint:


\begin{align}
	r_\mathcal{C} (t_\mathcal{X} - t_\mathcal{C}) = {r_\mathcal{C}}^\prime ({t_\mathcal{X}}^\prime - t_\mathcal{C}).
\end{align}


Ensure that $0 < {q_C}^\prime < 1$, and if the constraint is broken then reject the proposal. 



\textul{\textit{Step 3}}. Finally, in order to calculate the Metropolis-Hastings-Green ratio, return the determinant of the Jacobian matrix:



\begin{align}
	J &= \begin{bmatrix} \frac{\partial {t_\mathcal{X}}^\prime}{\partial t_\mathcal{X}} & \frac{\partial {t_\mathcal{X}}^\prime}{\partial q_\mathcal{L}} & \frac{\partial {t_\mathcal{X}}^\prime}{\partial q_\mathcal{R}} \\
						\frac{\partial {q_\mathcal{L}}^\prime}{\partial t_\mathcal{X}} & \frac{\partial {q_\mathcal{L}}^\prime}{\partial q_\mathcal{L}} & \frac{\partial {q_\mathcal{L}}^\prime}{\partial q_\mathcal{R}} \\
						\frac{\partial {q_\mathcal{R}}^\prime}{\partial t_\mathcal{X}} & \frac{\partial {q_\mathcal{R}}^\prime}{\partial q_\mathcal{L}} & \frac{\partial {q_\mathcal{R}}^\prime}{\partial q_\mathcal{R}} \end{bmatrix} \nonumber \\
		&= \begin{bmatrix} \frac{\partial {t_\mathcal{X}}^\prime}{\partial t_\mathcal{X}} & 0 & 0 \\
						\frac{\partial {q_\mathcal{L}}^\prime}{\partial t_\mathcal{X}} & \frac{\partial {q_\mathcal{L}}^\prime}{\partial q_\mathcal{L}} & 0\\
						\frac{\partial {q_\mathcal{R}}^\prime}{\partial t_\mathcal{X}} & 0 & \frac{\partial {q_\mathcal{R}}^\prime}{\partial q_\mathcal{R}} \end{bmatrix}.
\end{align}



As $J$ is triangular, its determinant $|J|$ is equal to the product of diagonal elements:


\begin{align}
	\ln |J| =&  \ln \{ \frac{\partial {t_\mathcal{X}}^\prime}{\partial t_\mathcal{X}} \times \frac{\partial {q_\mathcal{L}}^\prime}{\partial q_\mathcal{L}} \times \frac{\partial {q_\mathcal{R}}^\prime}{\partial q_\mathcal{R}} \} \nonumber \\
			=& \ln \frac{\partial {t_\mathcal{X}}^\prime}{\partial t_\mathcal{X}} +  \ln \frac{\partial {q_\mathcal{L}}^\prime}{\partial q_\mathcal{L}} + \ln \frac{\partial {q_\mathcal{R}}^\prime}{\partial q_\mathcal{R}} \nonumber \\
			=& \ln 1  \nonumber\\
			&+ \ln \; D \hat{F}\Big( \frac{t_\mathcal{X} - t_\mathcal{L}}{{t_\mathcal{X}}^\prime - t_\mathcal{L}} \times \hat{F^{-1}}(q_\mathcal{L}) \Big) + \ln \frac{\partial}{\partial q_\mathcal{L}} \frac{t_\mathcal{X} - t_\mathcal{L}}{{t_\mathcal{X}}^\prime - t_\mathcal{L}} \hat{F^{-1}} \Big( q_\mathcal{L} \Big) \nonumber\\
			&+ \ln \; D \hat{F}\Big( \frac{t_\mathcal{X} - t_\mathcal{R}}{{t_\mathcal{X}}^\prime - t_\mathcal{R}}\times \hat{F^{-1}}(q_\mathcal{R}) \Big) + \ln \frac{\partial}{\partial q_\mathcal{R}} \frac{t_\mathcal{X} - t_\mathcal{R}}{{t_\mathcal{X}}^\prime - t_\mathcal{R}} \hat{F^{-1}} \Big( q_\mathcal{R} \Big) \nonumber\\
			=& \ln \; D \hat{F}\Big( \frac{t_\mathcal{X} - t_\mathcal{L}}{{t_\mathcal{X}}^\prime - t_\mathcal{L}} \times \hat{F^{-1}}(q_\mathcal{L}) \Big) + \ln \; D \hat{F^{-1}} \Big( q_\mathcal{L} \Big) + \ln \frac{t_\mathcal{X} - t_\mathcal{L}}{{t_\mathcal{X}}^\prime - t_\mathcal{L}}\nonumber \\
			&+ \ln \; D \hat{F}\Big( \frac{t_\mathcal{X} - t_\mathcal{R}}{{t_\mathcal{X}}^\prime - t_\mathcal{R}}\times \hat{F^{-1}}(q_\mathcal{R}) \Big) + \ln \; D \hat{F^{-1}} \Big( q_\mathcal{R} \Big) + \ln  \frac{t_\mathcal{X} - t_\mathcal{R}}{{t_\mathcal{X}}^\prime - t_\mathcal{R}}.
\end{align}


As its final step, the operator returns $\ln |J|$. 


\subsection*{Small Pulley}

Just like the previous operator, \texttt{SmallPulley} operates on the root. 
Let $\mathcal{X}$ be the root node and let $\mathcal{L}$ and $\mathcal{R}$ be its two children.
However, unlike \texttt{SimpleDistance}, this operator alters the two genetic distances $d_\mathcal{L} = r_\mathcal{L} (t_\mathcal{X} - t_\mathcal{L}) = \hat{F^{-1}} (q_\mathcal{L}) (t_\mathcal{X} - t_\mathcal{L})$ and $d_\mathcal{R} = r_\mathcal{R} (t_\mathcal{X} - t_\mathcal{R}) = \hat{F^{-1}} (q_\mathcal{R}) (t_\mathcal{X} - t_\mathcal{R})$, while conserving their sum $d_\mathcal{L} + d_\mathcal{R}$.



\textul{\textit{Step 1}}. Propose new genetic distances for $d_\mathcal{L}$ and $d_\mathcal{R}$:

\begin{align}
	{d_\mathcal{L}}^\prime & \leftarrow d_\mathcal{L} + s\Sigma \\
	{d_\mathcal{R}}^\prime & \leftarrow d_\mathcal{R} - s\Sigma
\end{align}


Ensure that $0 < {d_\mathcal{L}}^\prime < d_\mathcal{L} + d_\mathcal{R}$, and if the constraint is broken then reject the proposal.  \\




\textul{\textit{Step 2}}. Propose new rate quantiles for the two children $\mathcal{L}$ and $\mathcal{R}$:


\begin{align}
	{q_\mathcal{L}}^\prime &\leftarrow \hat{F}\Big( \frac{{d_\mathcal{L}}^\prime}{t_\mathcal{X} - t_\mathcal{L}} \big) \nonumber\\
				 &\leftarrow \hat{F}\Big( \frac{\hat{F^{-1}} (q_\mathcal{L}) (t_\mathcal{X} - t_\mathcal{L}) + \Sigma}{t_\mathcal{X} - t_\mathcal{L}} \big) \\
	{q_\mathcal{R}}^\prime &\leftarrow \hat{F}\Big( \frac{{d_\mathcal{R}}^\prime}{t_\mathcal{X} - t_\mathcal{R}} \big) \nonumber \\
				 &\leftarrow \hat{F}\Big( \frac{\hat{F^{-1}} (q_\mathcal{R}) (t_\mathcal{X} - t_\mathcal{R}) - \Sigma}{t_\mathcal{X} - t_\mathcal{R}} \big).
\end{align}



\textul{\textit{Step 3}}. Return the determinant of the Jacobian matrix:


\begin{align}
	J &= \begin{bmatrix} \frac{\partial {q_\mathcal{L}}^\prime}{\partial q_\mathcal{L}} & \frac{\partial {q_\mathcal{L}}^\prime}{\partial q_\mathcal{R}} \\
						 \frac{\partial {q_\mathcal{R}}^\prime}{\partial q_\mathcal{L}} & \frac{\partial {q_\mathcal{R}}^\prime}{\partial q_\mathcal{R}} \end{bmatrix} \nonumber \\
	 &= \begin{bmatrix} \frac{\partial {q_\mathcal{L}}^\prime}{\partial q_\mathcal{L}} & 0 \\
						 0 & \frac{\partial {q_\mathcal{R}}^\prime}{\partial q_\mathcal{R}} \end{bmatrix}.
\end{align}


As $J$ is triangular/diagonal, its determinant $|J|$ is equal to the product of diagonal elements:



\begin{align}
	\ln |J| =&  \ln \{ \frac{\partial {q_\mathcal{L}}^\prime}{\partial q_\mathcal{L}} \times \frac{\partial {q_\mathcal{R}}^\prime}{\partial q_\mathcal{R}} \} \nonumber \\
			=& \ln  \frac{\partial {q_\mathcal{L}}^\prime}{\partial q_\mathcal{L}} + \ln \frac{\partial {q_\mathcal{R}}^\prime}{\partial q_\mathcal{R}}\nonumber \\
			=& \ln \; D \hat{F}\Big( \frac{\hat{F^{-1}} (q_\mathcal{L}) (t_\mathcal{X} - t_\mathcal{L}) + \Sigma}{t_\mathcal{X} - t_\mathcal{L}} \big) + \ln \frac{\partial {q_\mathcal{L}}^\prime}{\partial q_\mathcal{L}}  \frac{\hat{F^{-1}} (q_\mathcal{L}) (t_\mathcal{X} - t_\mathcal{L}) + \Sigma}{t_\mathcal{X} - t_\mathcal{L}} \nonumber\\
			&+ \ln \; D \hat{F}\Big( \frac{\hat{F^{-1}} (q_\mathcal{R}) (t_\mathcal{X} - t_\mathcal{R}) - \Sigma}{t_\mathcal{X} - t_\mathcal{R}} \big) + \ln \frac{\partial {q_\mathcal{R}}^\prime}{\partial q_\mathcal{R}}  \frac{\hat{F^{-1}} (q_\mathcal{R}) (t_\mathcal{X} - t_\mathcal{R}) - \Sigma}{t_\mathcal{X} - t_\mathcal{R}} \nonumber\\
			=& \ln \; D \hat{F}\Big( \frac{\hat{F^{-1}} (q_\mathcal{L}) (t_\mathcal{X} - t_\mathcal{L}) + \Sigma}{t_\mathcal{X} - t_\mathcal{L}} \big) + \ln \; D \hat{F^{-1}} (q_\mathcal{L}) \nonumber \\
			&+ \ln \; D \hat{F}\Big( \frac{\hat{F^{-1}} (q_\mathcal{R}) (t_\mathcal{X} - t_\mathcal{R}) - \Sigma}{t_\mathcal{X} - t_\mathcal{R}} \big) + \ln \; D\hat{F^{-1}} (q_\mathcal{R}) .
\end{align}


Thus, as its final step, the operator returns $\ln |J|$.



\clearpage
\section{CisScale operator}

\texttt{CisScale} was originally introduced by Zhang and Drummond 2020 for the \textit{real} parameterisation (therein named ucldstdevScaleOperator).
Under the \textit{quant} configuration, the \texttt{CisScale} operator works as follows. \\

\textul{\textit{Step 1}}. Propose a new value for the relaxed clock standard deviation $\sigma$


\begin{align}
	\sigma^\prime \leftarrow \sigma \times e^{s\Sigma}.
\end{align}


\textul{\textit{Step 2}}. Recalculate all branch substitution rate quantiles $q$ such that their rates $r$ remain constant


\begin{align}
	\text{let } & r = \hat{F}^{-1}(q|\sigma) \\ 
	\text{let } & r^\prime = r\\ 
	& q^\prime \leftarrow \hat{F}( r^\prime|\sigma^\prime) = \hat{F}( \hat{F}^{-1}(q|\sigma)|\sigma^\prime).
\end{align}


\textul{\textit{Step 3}}. Return the log Hastings-Green ratio of this proposal. 
If $\Sigma$ was drawn from a symmetric proposal kernel (such as the Bactrian distribution), this is equal to:

\begin{align}
	|J| &= \log(e^{s\Sigma}) + \log\Big( \frac{\delta }{\delta q} \hat{F}( \hat{F}^{-1}(q|\sigma)|\sigma^\prime) \Big) \\
	&= s\Sigma + \log D\hat{F}( \hat{F}^{-1}(q|\sigma)|\sigma^\prime) + \log D\hat{F}^{-1}(q|\sigma),
\end{align}

where derivatives $D\hat{F}$ and $D\hat{F}^{-1}$ can be approximated using either the piecewise linear model or standard numerical libraries. 

\clearpage
\section{Narrow exchange rates}


The \texttt{NarrowExchangeRate} operator is also compatible with rate quantiles.
This operator behaves the same as presented in the main article however the Hastings-Green ratio requires further augmentation due to changes in dimension throughout the proposal. \\


\textul{\textit{Step 1}}. Apply \texttt{NarrowExchange} to the current tree topology as described in the main article. 
This will return a Hastings ratio $H$ due to the asymmetry of this proposal.  \\


\textul{\textit{Step 2}}. Compute the relevant branch rates $r_i$ for $r \in \{A,B,C,D \}$ of the current state from their respective quantile parameters.  \\

\begin{align}
	r_i = \hat{F}^{-1}(q_i).
\end{align}


\textul{\textit{Step 3}}. Propose new rates and node heights and compute the Hastings-Green ratio of the real-space component of the proposal (e.g. Algorithms 1-2 of the main article).    \\


\begin{align}
	(r_A^\prime, r_B^\prime, r_C^\prime, r_D^\prime, t_D^\prime, |J_r|) \leftarrow \texttt{PROPOSAL}(r_A, r_B, r_C, r_D, t_D).
\end{align}


\textul{\textit{Step 4}}. Transform the rates back into quantiles.

\begin{align}
	q_i^\prime = \hat{F}(r_i^\prime).
\end{align}


\textul{\textit{Step 5}}. Compute the log Hastings-Green ratio of the interconversion between rates and quantiles.


\begin{align}
	\log|J_q| = \log \hat{F}(q) + \log \hat{F}^{-1}(r^\prime).
\end{align}

% double logHR = Math.log(pld.getDerivativeAtQuantile(qOld));
% logHR +=  Math.log(pld.getDerivativeAtQuantileInverse(rNew, qNew));


\textul{\textit{Step 6}}. Return the total log Hastings-Green ratio of this proposal: $\log H + \log|J_r| + \log|J_q|$.

\bibliography{references}




\end{document}


